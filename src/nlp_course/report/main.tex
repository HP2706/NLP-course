\documentclass{article}

\usepackage{setspace}
\usepackage{graphicx}
\usepackage{listings}
\usepackage{xcolor}
\usepackage{subcaption}
\usepackage[utf8]{inputenc}
\usepackage[T1,T2A]{fontenc}
\usepackage[T2A]{fontenc}
\usepackage[utf8]{inputenc}
\usepackage[russian]{babel}
\usepackage{polyglossia}   
\setmainlanguage{english}  
\setotherlanguage{russian}
\setmainfont{Times New Roman}
\setromanfont{Times New Roman} 
\usepackage{xeCJK}

\usepackage{fontspec}
\setmainfont{DejaVu Serif}

% Add this line to define \textrussian command
\newcommand{\textrussian}[1]{\foreignlanguage{russian}{#1}}

\lstset{
    language=Python,
    breaklines=true,
    basicstyle=\ttfamily,
    keywordstyle=\color{blue},
}

\begin{document}

% For Japanese text, wrap it in CJK environment
% \begin{CJK}{UTF8}{min}
% % Your content here
% \end{CJK}

\title{NLP Course Report}
\author{Hans Peter Lyngsøe, pvr448}

\maketitle

\section{Week 1}
% (a) Explore the dataset from https://huggingface.co/datasets/coastalcph/tydi_xor_rc. 
% Familiarize yourself with the dataset card, download the dataset and explore its columns. 
% Summarize basic data statistics for training and validation data in each of the languages 
% Finnish (fi), Japanese (ja) and Russian (ru).
\begin{enumerate}
    \item[(a)] 

    Basic statistics:
    \begin{itemize}
        \item the data is quite evenly distributed across the 3 languages.
        \item We note that there are more answerable than unanswerable by a factor of 10-1.
        \item train\_set: 15326
        \item val\_set: 3028
    \end{itemize}

    \begin{figure}[h]
        \centering
        \includegraphics[width=0.8\textwidth]{../week1/plots/week1_a_dataset.png}
        \caption{Distribution of labels in the dataset}
        \label{fig:label_distribution}
    \end{figure}

    \begin{figure}[h]
        \centering
        \includegraphics[width=0.8\textwidth]{../week1/plots/week1_a_lang_token_distribution_normalized.png}
        \caption{normalized Histogram for token count(using llama3 tokenizer) of answerable/unanswerable questions in the dataset}
        \label{fig:language_distribution}
    \end{figure}

    % (b) For each of the languages Finnish, Japanese and Russian, report the 5 most common 
    % words in the questions from the training set. What kind of words are they?
    \item[(b)] 

    In order to get a faitful representation of the most meaningful words, we opt to filter out stopwords arguing that these words are not meaningful and does not carry information about the question.
    We use the NLP library spacy that provides and index of stopwords for each language.

    we use the fact that both finnish and russian uses spaces to seperate their words, and thus we can extract distinct words using the space as delimiter.

    Japanese is somewhat different as a language, and it is unclear whether naive splitting on space is meaningfull. 
    Thus we use a japanese speific tokenizer to parse sentences into words.

    \begin{lstlisting}[language=Python]
    import MeCab
    def get_top_words(df: pd.DataFrame, lang: str, n=5):
        df_lang = df[df['lang'] == lang].copy()
        
        if lang == 'ja':
            mecab = MeCab.Tagger("-Owakati")  # Initialize MeCab tokenizer
            df_lang.loc[:, 'words_question_tokens'] = df_lang['question'].apply(lambda x: mecab.parse(x).split())
        else:
            df_lang.loc[:, 'words_question_tokens'] = df_lang['question'].apply(lambda x: x.split(' '))
        
        all_tokens = np.concatenate(df_lang['words_question_tokens'].values)
        unique, counts = np.unique(all_tokens, return_counts=True)
        sorted_indices = np.argsort(counts)[::-1]
        top_unique_tokens = unique[sorted_indices][:n]
        top_tokens_dict = {token: int(count) for token, count in zip(top_unique_tokens, counts[sorted_indices][:n])}
        return top_tokens_dict
    \end{lstlisting}


    we get the following results:

    \begin{figure}[h]
        \centering
        \begin{subfigure}[b]{0.3\textwidth}
            \centering
            \includegraphics[width=\textwidth]{../week1/plots/week1_b_top_5_tokens_fi.png}
            \caption{Finnish}
            \label{fig:top_5_tokens_fi}
        \end{subfigure}
        \hfill
        \begin{subfigure}[b]{0.3\textwidth}
            \centering
            \includegraphics[width=\textwidth]{../week1/plots/week1_b_top_5_tokens_ja.png}
            \caption{Japanese}
            \label{fig:top_5_tokens_ja}
        \end{subfigure}
        \hfill
        \begin{subfigure}[b]{0.3\textwidth}
            \centering
            \includegraphics[width=\textwidth]{../week1/plots/week1_b_top_5_tokens_ru.png}
            \caption{Russian}
            \label{fig:top_5_tokens_ru}
        \end{subfigure}
        \caption{Top 5 tokens in Finnish, Japanese, and Russian}
        \label{fig:top_5_tokens_all}
    \end{figure}

    NOTE: todo use an embedding model to get a sense for the meaning of the words, consider how you do tokenization and 
    what the implications are. what are words and what are stopwords

% (c) Implement a rule-based classifier that predicts whether a question is answerable 
% or impossible, only using the document (context) and question. You may use machine 
% translation as a component. Use the answerable field to evaluate it on the validation set. 
% What is the performance of your classifier for each of the languages Finnish, Japanese and Russian?
    \item[(c)] 
\end{enumerate}

\section{Week 37 (9--15 September)}
% Let k be the number of members in your group (k ∈ {1, 2, 3}). Implement k different 
% language models for the questions in the three languages Finnish, Japanese and Russian, 
% as well as for the document contexts in English (total k × 4 language models), using the 
% training data. Evaluate each of them on the validation data, report their performance 
% and discuss the results.

\section{Week 38 (16--22 September)}
% Let k be the number of members in your group. For each of the three languages Finnish, 
% Japanese and Russian separately, using the training data, train k different classifiers 
% that receive the document (context) and question as input and predict whether the question 
% is answerable or impossible given the context. Evaluate the classifiers on the respective 
% validation sets, report and analyse the performance for each language and compare the 
% scores across languages.

\section{Week 39 (23--29 September)}
% Let k be the number of members in your group. Using the training data in Finnish, Japanese 
% and Russian separately, train k different sequence labellers, which predict the tokens in 
% a document context that constitute the answer to the corresponding question. You can decide 
% whether to train one model per language or a single model for all three languages. Evaluate 
% using a sequence labelling metric on the validation set, report and analyse the performance 
% for each language and compare the scores across languages.

\section{Week 40 (30 September--6 October)}
% Use the subset of the questions in Finnish, Japanese and Russian to train (or fine-tune) 
% an encoder-decoder model that receives the question and context as input and generates 
% the in-language answer. You can decide whether to train one model per language or a 
% single model for all three languages.

\section{Week 41+ (from 7 October)}
% Use all questions in Finnish, Japanese and Russian to train (or fine-tune) an encoder-decoder 
% model that receives the question and context as input and generates the English answer. 
% You can decide whether to train one model per question language or a single model for all 
% three languages. Evaluate using a text generation metric on the validation set, and compare 
% the overall results between answerable and unanswerable examples.

\end{document}